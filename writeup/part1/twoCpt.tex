
As a starting example, we consider a compartment pharmacokinetic model for a single patient.
The patient receives multiple doses at regular time intervals and the drug plasma concentration is recorded over time.
Our goal is to infer the physiological parameters of the patient, pertinent to the drug's pharmacokinetics, and the variance of our measurements.

\subsection{Pharmacokinetic model and clinical event schedule} 

The two compartment pharmacokinetic model describes how the drug circulates in the patient's body, until it is cleared out (Figure~\ref{fig:twocpt}).
The drug is orally administered and therefore enters the system through the gut.
Over time it is cleared out.
Once the drug is introduced in the system, the \textit{natural evolution} of the system is described by a system of ODEs.
In the case of a two compartment model with a first-order absorption from the gut, the system is the following:
%
\begin{eqnarray*}
  \frac{\mathrm d y_\mathrm{gut}}{\mathrm d t} & = & - k_a y_\mathrm{gut} \\ 
  \frac{\mathrm d y_\mathrm{central}}{\mathrm d t} & = & k_a y_\mathrm{gut} - \left (\frac{CL}{V_\mathrm{cent}} + \frac{Q}{V_\mathrm{cent}} \right) y_\mathrm{cent} + \frac{Q}{V_\mathrm{peri}} y_\mathrm{peri} \\
  \frac{\mathrm d y_\mathrm{peri}}{\mathrm d t} & = & \frac{Q}{V_\mathrm{cent}} y_\mathrm{cent} - \frac{Q}{V_\mathrm{peri}} y_\mathrm{peri}
\end{eqnarray*}
%
where...


\subsection{Specifying a model in Stan}

We can specify a model using a Stan file.
From R, we then run inference algorithms which take this Stan file as an input.

To define a model, we need a procedure which returns the log joint distribution, $\log p(\mathcal D, \theta)$.
Our first task is to declare data, $\mathcal D$, and parameter, $\theta$, variables, using the coding blocks \texttt{data} and \texttt{parameters}.

\begin{lstlisting}[style=stan, numbers=none] 
  data {
  
  }
 
\end{lstlisting}
