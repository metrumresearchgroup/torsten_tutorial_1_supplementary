
Bayesian inference offers a principled approach to learn about unknowns from data using a probabilistic analysis.
The conclusions we draw are based on the posterior distribution which, in all but the simplest cases, is intractable.
We can however probe the posterior and numerically compute its characteristics, e.g. expectation value, variance, and quantiles, by using algorithms such as Markov chains Monte Carlo (MCMC).
Implementing these algorithms is a tedious and error prone endeavor, but fortunately modelers can often rely on existing software with efficient implementations.
Probabilistic programing languages are designed to let users specify a model and then run inference on the backend.

In the field of pharmacometrics, statistical software such as NONMEM \cite{author:0000} and Monolix \cite{author:0000} support many routines to specify and analyze pharmacokinetic and pharmacodynamic population models.
There also exist a host of more general probabilistic languages such as, to only name a few, BUGS \cite{author:0000} and more recently Stan \cite{Carpenter:2017}, which will be the focus of this tutorial.
Stan supports a rich library of probability densities, matrix operations and numerical solvers for differential equations.
These features make for a rich and flexible language, however specifying common pharmacometrics models can be tedious.
Torsten extends Stan by providing a suite of functions to facilitate the specification of pharmacometrics models.
These functions make it straightforward to model the event schedule of a clinical trial and parallelize computation across patients, and also provide analytical solutions for ordinary differential equations (ODEs) which commonly arise in the field.


\subsection{Why Stan?}

We believe that Stan, coupled with Torsten, can be an important addition to the pharmacometrician's toolkit, especially for Bayesian data analysis.

The most obvious strength of Stan is its flexibility: it is straightforward to specify priors, systems of ODEs, a broad range of measurement models, missing data models and hierarchies.
Because of this flexibility, various data sources and their corresponding measurement models can be combined into one large model, over which full Bayesian inference can be performed \cite[e.g][]{Weber:2018}.

Secondly, Stan offers efficient inference algorithms, most notably its adaptive \textit{Hamiltonian Monte Carlo} (HMC) sampler, a gradient-based Markov chains Monte Carlo algorithm which scales in high dimensions \cite{Betancourt:2018, Hoffman:2014}, automatic differentiation variational inference \cite{author:0000}, and penalized maximum likelihood estimators.
Stan's inference algorithms are supported by a state-of-the-art automatic differentiation library, which efficiently generates the requisite derivatives \cite{Carpenter:2015}.

Thirdly, Stan runs many diagnostics -- including the detection of divergent transitions \cite{Betancourt:2018}, and the improved computation of effective sample sizes and scale reduction factors, $\hat R$ \cite{Vehtari:2020}, and more -- and gives detailed warning messages.
This makes it considerably easier to identify issues with our inference and our models.
Several of these tools improve commonly used diagnostics which may not detect important problems, in which case our model fitting fails without us realizing it.
Stan fails better: it fails loudly. 

Last but not least, both Stan and Torsten are open-source projects, meaning  they are free and their source code can be examined, and if needed scrutanized.
The projects are under active development with new features being continuously added.
%Stan's vibrant community readily assists both new and experienced users, through the Stan forum: \url{https://discourse.mc-stan.org/}.

\subsection{Bayesian inference: notation, goals, and comments}

Given observed data, $\mathcal D$, and latent variables, $\theta$, a Bayesian model is defined by the joint distribution, $p(\mathcal D, \theta)$.
The latent variables can include model parameters, missing data, and more.
In this tutorial, we are mostly concern with estimating model parameters.
The Bayesian proposition is to base our inference on the \textit{posterior}, $p(\theta \mid \mathcal D)$.
This posterior density describes the set of parameter values which are consistent with the model and the data, with the more consistent parameter values being assigned a higher density.
One interpretation is that we are treating the parameters as ``random variables''.
Technically this is true if we go back to the measure theoretical definition of a random variable. 
In a more colloquial setting, this turn of phrase may be more suggestive than needed.
Our interpretation is the following: we want to describe all the reasonable parameter values, quantify how consistent they are with our model and data, and to do this, the right mathematical formalism is that of a probability distribution.

It is worth pointing out that the posterior density is an unfathomable object which lives in a high dimensional space.
There is no such thing as ``computing the posterior distribution''.
We cannot even numerically evaluate the posterior density at any particular point!
Instead we must probe the posterior distribution and learn the characteristics that interest us the most.
In our experience, this often includes a measure of a central tendency and a quantification of uncertainty, for example the mean and the variance, or the median and several quantiles.
For skewed or multimodal distributions, we may want a more refine analysis which looks at many quantiles.
What we compute are estimates of these summarizing quantities.
One strategy is to generate \textit{approximate} samples from the posterior distribution and then use sample mean, sample variance, and sample quantiles as our estimators.

Bayes' rule teaches us that
\begin{equation*}
  p(\theta \mid \mathcal D) = \frac{p(\mathcal D, \theta)}{p(\mathcal D)}.
\end{equation*}
Typically we can evaluate the joint density -- as this is how we define our model! -- but not the denominator.
Lucky for us, HMC only requires the log joint density and its gradient to generate approximate samples from the posterior distribution.
For this procedure to be well-defined, $\theta$ must be a continuous variable.
Discrete parameters require a special treatment, which we will not discuss in this tutorial.
A Stan program specifies a method to evaluate $\log p(\mathcal D, \theta)$.
Thanks to automatic differentiation, this implicitly defines a method to compute $\nabla_\theta \log p(\mathcal D, \theta)$.

The joint distribution observes a convenient decomposition,
\begin{equation*}
  p(\mathcal D, \theta) = p(\theta) p(\mathcal D \mid \theta),
\end{equation*}
%
with $p(\theta)$ the \textit{prior} distribution and $p(\mathcal D \mid \theta)$ the \textit{likelihood}.
The prior encodes information about the parameters, usually based on scientific expertise or results from previous analysis.
It is a potent tool to encode prior knowledge -- hence the name -- or lack thereof using a probabilistic statement.
The likelihood tells us how the data is distributed for a fixed parameter value and, per one interpretation, can be thought of as a ``story of how the data is generated'' \cite{Gelman:2013}.

\subsection{Bayesian workflow}

Bayesian inference is only one step of a broader modeling process, which we might call the Bayesian workflow \cite{Betancourt:2018, Gelman:2020}.
Once we fit the model, we need to check the inference and if needed, fine tune our algorithm, or potentially change method.
And once we trust the inference, we naturally need to check the fitted model.
Our goal is to understand the shortcomings of our model and motivate useful revisions.
During the early stages of model development, this mostly comes down to troubleshooting our implementation and later this ``criticism'' step can lead to deeper insights.

All through the tutorial, we will demonstrate how Stan and Torsten can be used to check our inference and our fitted model.

\subsection{Setting up Stan and Torsten}

Detailed instructions on installing Stan and Torsten can be found on \url{https://github.com/metrumresearchgroup/Torsten}.
At its core, Stan is a C++ library but it can be interfaced with one of many scripting languages, including R, Python, and Julia.
We will use cmdStanR, which is a lightweight wrapper of Stan in R.
While RStan is compatible with Torsten and a popular option for R users, its development lags a little behind that of Stan.
In an addition, we use the packages Posterior \cite{author:0000} and BayesPlot \cite{author:0000}.

The R and Stan code for all the examples can be at \url{link}.

 \subsection{Prerequisites and resources}
 
 Our aim is to provide a self-contained discussion, while trying to remain concise.
 We assume the reader is familiar with compartment models as they arise in pharmacokinetic and pharmacodynamic models, and has experience with data that describe a clinical event schedule.
 For the latter, we follow the convention established by NONMEM.
 Exposure to Bayesian statistics and inference algorithms is desirable, in particular an elementary understanding of standard MCMC methods.
 We expect the reader to know R but we don't assume any background in Stan.
 
 Helpful reads include the \textit{Stan User Manual} \cite{author:0000} and the \textit{Torsten User Manual} \cite{author:0000}.
 A comprehensive textbook on Bayesian modeling is \textit{Bayesian Data Analysis} by Gelman et al (2013) \cite{Gelman:2013b}, with more recent insights on the Bayesian workflow provided by Gelman et al (2020). 
 Betancourt (2018) offers an accessible discussion on MCMC methods, with an emphasis on HMC \cite{Betancourt:2018}.
 We will discuss in the conclusion additional readings which can complement this tutorial.
  
 
 
 