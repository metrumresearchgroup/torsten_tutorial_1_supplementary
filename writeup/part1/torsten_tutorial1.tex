\documentclass[11pt]{article}
\usepackage{newclude}
\usepackage{apacite}
\usepackage{amssymb,amsmath}
\usepackage{color}
\usepackage{graphicx}
\usepackage{natbib}

% TODO: change options
\usepackage{hyperref}

% Verbantim - use to define \texttt{Stan} Code
\usepackage{fancyvrb}

% Code and packages for code figures
\usepackage{pdfpages}
\usepackage{listings}
\newsavebox{\fmbox}
\newenvironment{fmpage}[1]
   {\begin{lrbox}{\fmbox}\begin{minipage}{#1}}
   {\end{minipage}\end{lrbox}\fbox{\usebox{\fmbox}}}

% enumerate using letters
\usepackage{enumitem}

% package to control position of figures
\usepackage{float}

% control margins
\usepackage[left=4cm, right=4cm, top=2cm]{geometry}

% for fancy Mike style table
\usepackage[table]{xcolor} 
\usepackage{multirow}

% algorithm
\usepackage{algorithm, setspace}
\usepackage[noend]{algpseudocode}
\DefineVerbatimEnvironment{Stancode}{Verbatim}{fontsize=\small,xleftmargin=2em}  % \texttt{Stan} Code

% package for drawing graphs
\usepackage{tikz}
\usetikzlibrary{positioning}

% quote
\usepackage{epigraph}

\title{Torsten Tutorial, Part I}
\author{}
\date{}

\begin{document}

\maketitle

\begin{abstract}
  Stan is an open-source probabilistic programing language, primarily designed to do Bayesian data analysis.
  Its main inference algorithm is an adaptive Hamiltonian Monte Carlo sampler, supported by state of the art gradient computation.
  Stan bolsters several strengths, notably efficient computation, an expressive language which offers a great deal of flexibility, and numerous diagnostics that allow the modeler to check whether the inference is reliable.
  Over the past years, much development efforts have been geared towards supporting models based on differential equations, such as those that arise in pharmacometrics.
  Torsten extends Stan with a suite of functions that facilitate the specification of pharmacokinetic and pharmacodynamic models, and makes it straightforward to specify a clinical event schedule.
  Part I of this tutorial demonstrates how to build, fit, and criticize simple pharmacokinetic models using Stan and Torsten.
\end{abstract}

\section{Introduction}
\input{Intro.tex}
% Basic concepts of Bayesian modeling
% Why Stan?
% Why Torsten?
% Downloading the software and references.

\section{Two compartment model}

As a starting example, we demonstrate the analysis of longitudinal plasma drug concentration data from a single individual using a linear 2 compartment model with first order absorption.
The individual receives multiple doses at regular time intervals and
the plasma drug concentration is recorded over time.
Our goal is to estimate the posterior distribution of the parameters of the model describing the time course of the plasma drug concentrations in this individual.

\subsection{Pharmacokinetic model and clinical event schedule} \label{sec:twocpt}

Let's suppose an individual receives a 1,200 mg dose of a drug every
12 hours, until they have received a total of 14 doses. Drug
concentrations are measured in plasma obtained from blood sampled at
0.083, 0.167, 0.25, 0.5, 0.75, 1, 1.5, 2, 3, 4, 6 and 8 hours
following the first, second and last dose, at the time of all
remaining doses. and at 12, 18 and 24 hours following the last dose.  

The two compartment pharmacokinetic model describes how the drug circulates in the individual's body, until it is cleared out (Figure~\ref{fig:twocpt}).
The drug is orally administered and enters the system through the gut.
Once the drug is introduced in the system, the \textit{natural evolution} of the system is described by a system of ODEs.
In the case of a two compartment model with a first-order absorption from the gut, the system is the following:
%
\begin{subequations}
\begin{eqnarray*}
  \frac{\mathrm d y_\mathrm{gut}}{\mathrm d t} & = & - k_a y_\mathrm{gut} \\ 
  \frac{\mathrm d y_\mathrm{cent}}{\mathrm d t} & = & k_a y_\mathrm{gut} - \left (\frac{CL}{V_\mathrm{cent}} + \frac{Q}{V_\mathrm{cent}} \right) y_\mathrm{cent} + \frac{Q}{V_\mathrm{peri}} y_\mathrm{peri} \\
  \frac{\mathrm d y_\mathrm{peri}}{\mathrm d t} & = & \frac{Q}{V_\mathrm{cent}} y_\mathrm{cent} - \frac{Q}{V_\mathrm{peri}} y_\mathrm{peri}
\end{eqnarray*}
\label{eq:twocpt}
\end{subequations}
%
with
\begin{itemize}
  \setlength\itemsep{0em}
  \item $y(t)$: the drug mass in each compartment (mg),
  \item $k_a$: the rate constant at which the drug flows from the gut to the central compartment (h$^{-1}$),
  \item $Q$: the clearance at which the drug flows back and forth between the central and the peripheral compartment (L/h),
  \item $CL$: the clearance at which the drug is cleared from the central compartment (L / h),
  \item $V_\mathrm{cent}$: the volume of the central compartment (L),
  \item $V_\mathrm{peri}$: the volume of the peripheral compartment (L).
\end{itemize}

\begin{figure}
  \begin{center}
  \includegraphics[width=5in]{../figures/TwoCptNice.png}
  \caption{Two compartment model with first-order absorption from the gut.}
  \label{fig:twocpt}
  \end{center}
\end{figure}

During the trial, the patient receives a dose of 1,200 mg every 12 hours, until they have received a total of 14 doses.
Measurements are taken shortly after the first, second, and last doses, and at regular intervals during the treatment.
Both intervention and measurement events are described by the event schedule, which follows the convention established by NONMEM.
This means the user must provide for each event the following variables: \texttt{cmt}, \texttt{evid}, \texttt{addl}, \texttt{ss}, \texttt{amt}, \texttt{time}, \texttt{rate}, and \texttt{ii}.
Table~\ref{tab:event_schedule} provides an overview of the roll of each variable and more details can be found in the \textit{Torsten User Manual}.

\begin{table}
  \renewcommand{\arraystretch}{1.5}
  \begin{center}
  \begin{tabular} {l l l}
  \rowcolor[gray]{0.95} \textbf{Variable} & \textbf{Description} \\
  \texttt{cmt} & Compartment in which event occurs.\\
  \rowcolor[gray]{0.95} \texttt{evid} & Type of event: (0) measurement, (1) dosing. \\
  \texttt{addl} & For dosing events, number of additional doses.  \\
  \rowcolor[gray]{0.95} \texttt{ss} & Steady state indicator: (0) no, (1) yes. \\
  \texttt{amt} & Amount of drug administered. \\
  \rowcolor[gray]{0.95} \texttt{time} & Time of the event. \\
  \texttt{rate} & For dosing by infusion, rate of infusion. \\
  \rowcolor[gray]{0.95} \texttt{ii} & For events with multiple dosing, inter-dose interval.
  \end{tabular}
  \end{center}
  \caption{Available variables in Torsten to specify an event schedule.}
  \label{tab:event_schedule}
\end{table}

\subsection{Statistical model}

% We define a generative model.
Given a treatment, $x$, and the physiological parameters, $\{ k_a, Q, CL, V_\mathrm{cent}, V_\mathrm{peri} \}$, we compute the drug plasma concentration, $\hat c$, by solving the relevant ODEs.
Our measurements, $y$, are a perturbation of $\hat c$.
This is to account for the fact that our model is not perfect and that our measurements have finite precision.
We model this residual error using a lognormal distribution, that is
\begin{equation*}
  y \mid \hat c, \sigma \sim \mathrm{logNormal}(\log \hat c, \sigma^2),
\end{equation*}
where $\sigma$ is a scale parameter we wish to estimate.
The deterministic computation of $\hat c$ along with the measurement model, defines our likelihood function $p(y \mid \theta, x)$, where $\theta = \{ k_a, Q, CL, V_\mathrm{cent}, V_\mathrm{peri}, \sigma \}$.

It remains to define a prior distribution, $p(\theta)$.
Our prior should allocate probability mass to every plausible parameter value and exclude patently absurd values.
For example the volume of the central compartment is on the order of ten liters, but it cannot be the size of the Sun.
In this simulated example, our priors for the individual parameters are based on population estimates from previous (hypothetical) studies.
\begin{eqnarray*}
  CL & \sim & \mathrm{logNormal}(\log(10), 0.25);  \\
  Q & \sim & \mathrm{logNormal}(\log(15), 0.5); \\
  V_\mathrm{cent} & \sim & \mathrm{logNormal}(\log(35), 0.25); \\
  V_\mathrm{peri} & \sim & \mathrm{logNormal}(\log(105), 0.5); \\
  k_a & \sim & \mathrm{logNormal}(\log(2.5), 1); \\
  \sigma & \sim & \mathrm{Half-Normal}(0, 1); \\
\end{eqnarray*}

Suggestions for building priors can be found in references \cite{Gabry:2019, Betancourt:2020} and at \url{https://github.com/stan-dev/stan/wiki/Prior-Choice-Recommendations}.

\subsection{Specifying a model in Stan}

We can now specify our statistical model using a Stan file, which is divided into coding blocks, each with a specific role.
From R, we then run inference algorithms which take this Stan file as an input.

\subsubsection{Data and parameters block}

To define a model, we need a procedure which returns the log joint distribution, $\log p(\mathcal D, \theta)$.
Our first task is to declare the data, $\mathcal D$, and the parameters, $\theta$, using the coding blocks \texttt{data} and \texttt{parameters}.
It is important to distinguish the two.
The data is fixed.
By contrast, the parameter values change as HMC explores the parameter space, and gradients of the joint density are computed with respect to $\theta$, but not $\mathcal D$.

For each variable we introduce, we must declare a type and, for containers such as arrays, vectors, and matrices, the size of the container.
In addition, each statement ends with a semi-colon.
It is possible to specify constraints on the parameters, using the keywords \texttt{lower} and \texttt{upper}.
If one of these constraints is violated, Stan returns an error message.
More importantly, constrained parameters are transformed into unconstrained parameters -- for instance, positive variables are put on the log scale -- which greatly improves computation.

\begin{lstlisting}[style=stan, numbers=none] 
  data {
    int<lower = 1> nEvent;
    int<lower = 1> nObs;
    int<lower = 1> iObs[nObs];  // index of events which
                                // are observations.

    // Event schedule
    int<lower = 1> cmt[nEvent];
    int evid[nEvent];
    int addl[nEvent];
    int ss[nEvent];
    real amt[nEvent];
    real time[nEvent];
    real rate[nEvent];
    real ii[nEvent];

    // observed drug concentration
    vector<lower = 0>[nObs] cObs;
  }
  
  parameters {
    real<lower = 0> CL;
    real<lower = 0> Q;
    real<lower = 0> VC;
    real<lower = 0> VP;
    real<lower = 0> ka;
    real<lower = 0> sigma;
  }
\end{lstlisting}

\subsubsection{model block}

Next, the \texttt{model} block allows us to modify the variable \texttt{target}, which Stan recognizes as the log joint distribution.
The following statement increments \texttt{target} using the prior on $\sigma$, which is a normal density, truncated at 0 to only put mass on positive values.
\begin{lstlisting}
target += normal_lpdf(sigma | 0, 1);
\end{lstlisting}
The truncation is implied by the fact $\sigma$ is declared as lower-bounded by 0 in the parameters block.
An alternative syntax is the following:
\begin{lstlisting}
sigma ~ normal(0, 1);
\end{lstlisting}
This statement now looks like our statistical formulation and makes the code more readable.
But we should be mindful that this is not a sampling statement, rather instructions on how to increment \texttt{target}.
We now give the full model block:
\begin{lstlisting}
model {
  // priors
  CL ~ lognormal(log(10), 0.25); 
  Q ~ lognormal(log(15), 0.5);
  VC ~ lognormal(log(35), 0.25);
  VP ~ lognormal(log(105), 0.5);
  ka ~ lognormal(log(2.5), 1);
  sigma ~ normal(0, 1);

  // likelihood
  cObs ~ lognormal(log(concentrationHat[iObs]), sigma);
}
\end{lstlisting}
%
The likelihood statement involves a crucial term we have not defined yet: \texttt{concentrationHat}.
Additional variables can be created using the transformed data and transformed parameters blocks.
We will take advantage of these to compute the drug concentration in the central compartment for each event.
Note that for the likelihood, we only use the concentration during observation events, hence the indexing \texttt{[iObs]}.

\subsubsection{Transformed data and transformed parameters block} \label{sec:twocpt_transformed_parameters}

In \texttt{transformed data}, we can construct variables which only depend on the data.
For this model, we simply specify the number of compartments in our model (including the gut), \texttt{nCmt}, and the numbers of physiological parameters, \texttt{nTheta}, two variables which will come in handy shortly.
\begin{lstlisting}
  transformed data {
    int nCmt = 3;
    int nTheta = 5; 
  }
\end{lstlisting}
%
Because the data is fixed, this operation is only computed once.
By contrast, operations in the \texttt{transformed parameters} block need to be performed (and differentiated) for each new parameter values.

To compute \texttt{concentrationHat} we need to solve the relevant ODE within the clinical event schedule.
Torsten provides a function which returns the drug mass in each compartment at each time point of the event schedule.
\begin{lstlisting}
matrix<lower = 0>[nCmt, nEvent] 
  mass = pmx_solve_twocpt(time, amt, rate, ii, evid,
                          cmt, addl, ss, theta);
\end{lstlisting}
%
The first eight arguments define the event schedule and the last argument, \texttt{theta}, is an array containing the physiological parameters, and defined as follows:
\begin{lstlisting}
real theta[nTheta] = {CL, Q, VC, VP, ka};
\end{lstlisting}
%
It is also possible to have \texttt{theta} change between events, and specify lag times and bio-availibilities fractions, although we will not take advantage of these features in the example at hand.

The Torsten function we have chosen to use solves the ODEs analytically.
Other routines use a matrix exponential, a numerical solver, or a combination of analytical and numerical methods \cite{Margossian:2017}.
It now remains to compute the concentration in the central compartment at the relevant times.
The full \texttt{transformed parameters} block is as follows:
\begin{lstlisting}
transformed parameters {
  real theta[nTheta] = {CL, Q, VC, VP, ka};
  row_vector<lower = 0>[nEvent] concentrationHat;
  matrix<lower = 0>[nCmt, nEvent] mass;

  mass = pmx_solve_twocpt(time, amt, rate, ii, evid, 
                          cmt, addl, ss, theta);

  // Extract mass in central compartment and divide 
  // by central volume.
  concentrationHat = mass[2, ] ./ VC;
}  
\end{lstlisting} 

The Stan file contains all the coding blocks in the following order: \texttt{data}, \texttt{transformed data}, \texttt{parameters}, \texttt{transformed parameters}, \texttt{model}.
The full Stan code can be found in the Supplementary Material.

\subsection{Calling Stan from R}\label{sec:call_stan_from_R}
The package CmdStanR allows us to run a number of algorithms on a model defined in a Stan file.
An excellent place to get started with the package is \url{https://mc-stan.org/cmdstanr/articles/cmdstanr.html}.

The first step is to ``transpile'' the file -- call it \texttt{twocpt.stan} --, that is translate the file into C++ and then compile it.
\begin{lstlisting}
mod <- cmdstan_model("twocpt.stan")
\end{lstlisting}
%
We can then run Stan's HMC sampler by passing in the requisite data and providing other tuning parameters.
Here: (i) the number of Markov chains (which we run in parallel), (ii) the initial value for each chain, (iii) the number of warmup iterations, and (iv) the number of sampling iterations.
\begin{lstlisting}
fit <- mod$sample(data = data, chains = n_chains,
                  parallel_chains = n_chains,
                  init = init,
                  iter_warmup = 500, 
                  iter_sampling = 500)
\end{lstlisting}
%
There are several other arguments we can pass to the sampler and which we will take advantage of throughout the tutorial.
For applications in pharmacometrics, we recommend specifying the initial starting points via the \texttt{init} argument, as the defaults may not be appropriate.
In this tutorial, we draw the initial points from their priors by defining an appropriate R function.

The resulting \texttt{fit} object stores the samples generated by HMC from which can deduce the sample mean, sample variance, and sample quantiles of our posterior distribution.
This information is readily accessible using \texttt{fit\$summary()} and summarized in table~\ref{tab:summary}.
We could also extract the samples and perform any number of operations on them.

\begin{table}[!h]
  \renewcommand{\arraystretch}{1.5}
  \begin{tabular}{l l l l l l l l l l}
  \rowcolor[gray]{0.95} & \bf mean & \bf median & \bf sd & \bf mad & \bf q5 & \bf q95 & $\bf \hat R$ & \bf ESS$_\mathrm{bulk}$ & \bf ESS$_\mathrm{tail}$ \\
$CL$    &    10.0  &  10.0  &  0.378  & 0.367 &  9.39 &  10.6  &  1.00   & 1580  &  1348 \\
\rowcolor[gray]{0.95} $Q$      &  19.8    & 19.5  &  4.00  &  4.01 &  13.8  &   26.8 &   1.00  &   985 &    1235 \\
$V_\mathrm{cent}$   &   41.2  &  40.8 &   9.71   & 9.96  & 25.6 &   57.7   & 1.00   &  732  &  1120 \\
\rowcolor[gray]{0.95} $V_\mathrm{peri}$    &  124 &   123 &    18.0  &  18.0 &    97.1 &  155 &   1.00  &  1877 &    1279 \\
$k_a$       &  1.73   & 1.67 &  0.523  & 0.522   & 1.01 &   2.68 &   1.00   &  762 &    1108 \\
\rowcolor[gray]{0.95} $\sigma$   & 0.224 &   0.222 &  0.0244 &  0.0232 &  0.187 &   0.269 &  1.01  & 1549 &   1083
  \end{tabular}
  \caption{Summary of results when fitting a two compartment model. \textit{The first columns return sample estimates of the posterior mean, median, standard deviation, median absolute deviation, $5^\mathrm{th}$ and $95^\mathrm{th}$ quantiles, based on our approximate samples.
  The next three columns return the $\hat R$ statistics and the effective sample size for bulk and tail estimates, and can be used to identify problems with our inference.}}
  \label{tab:summary}
\end{table}

\subsection{Checking our inference}

Unfortunately there is no guarantee that a particular algorithm will work across all the applications we will encounter.
We can however make sure that certain necessary conditions do not break.
Much of the MCMC literature focuses on estimating expectation values, and we will use these results to develop some intuition.

\subsubsection{Central limit theorem}

Many common MCMC concepts, such as effective sample size, are amiable to intuitive interpretations.
But to really grasp their meaning and take advantage of them, we must examine them in the context of central limit theorems.

For any function $f$ of our latent parameters $\theta$, we define the posterior mean to be
\begin{equation*}
  \mathbb E f = \int_\Theta f(\theta) p(\theta \mid \mathcal D) \mathrm d \theta,
\end{equation*}
%
a quantity also termed the \textit{expectation value}.
If we were able to generate exact independent samples, 
\begin{equation*}
  \theta^{(1)}, \theta^{(2)}, ..., \theta^{(n)} \overset{\mathrm{i.i.d}}{\sim} p(\theta \mid \mathcal D),
\end{equation*}
one sensible estimator would be the sample mean,
\begin{equation*}
  \hat {\mathbb E} f = \frac{1}{n} \sum_{i = 1}^n f \left (\theta^{(i)} \right).
\end{equation*}
%
Then, provided the variance of $f$ is finite, the \textit{central limit theorem} teaches us that the sample mean converges, as the sample size increases, to a normal distribution.
We will not dwell on the technical details of the theorem and go straight to its practical implication, namely:
\begin{equation*}
  \hat {\mathbb E} f \overset{\mathrm{approx}}{\sim} \mathrm{Normal} \left ( \mathbb E f, \frac{\mathrm{Var} f}{n} \right ),
\end{equation*}
%
where the deviation from the approximating normal has order $\mathcal O(1 / n^2)$.
This means that even for a moderate sample size, the approximation tends to be very good.
This is a powerful result for two reasons: first it tells us that our estimator is unbiased and more importantly that the expected squared error is driven by the variance of $f$ divided by our sample size, $n$.

Unfortunately, estimates constructed with MCMC samples will, in general, neither be unbiased, nor will their variance decrease at rate $n$.
For our estimators to be useful, we must therefore check that our samples are unbiased and then use a corrected central limit theorem.

\subsubsection{Checking for bias with $\hat R$}

MCMC samples are biased for several reasons.
Perhaps the most obvious one is that Markov chains generate correlated samples, meaning any sample has some correlation with the initial point.
If we run the algorithm for enough iterations, the correlation to the initial point becomes negligible and the chain ``forgets'' its starting point.
But what constitutes enough iterations?
It isn't hard to construct examples where removing the initial bias in any reasonable time is a hopeless endeavor. 

To identify this bias, we run multiple Markov chains, each started at different points, and check that they all convergence to the same region of the parameter space.
More precisely, we shouldn't be able to distinguish the Markov chains based on the samples alone.
One way to check this is to compute the $\hat R$ statistics, for which we provide an intuitive definition:
\begin{equation*}
  \hat R \overset{\mathrm{intuitively}}{=} \frac{\mathrm{Between \ chain \ variance}}{\mathrm{Within \ chain \ variance}}.
\end{equation*}
%
If the chains are mixing properly, then $\hat R \approx 1.0$, as is the case in table~\ref{tab:summary}.
Stan uses an improved $\hat R$ statistics described in a recent paper by \citet{Vehtari:2020}.
We can also visually check that the chains are properly mixing using a trace plot (Figure~\ref{fig:trace}).

\begin{figure}
  \begin{center}
  \includegraphics[width = 6in]{../figures/twocpt_traceplots_4x8.pdf}
  \caption{Trace plots. \textit{The sampled values for each parameters are plotted against the iterations during the sampling phase. Multiple Markov chains were initialized at different points. However, once in the sampling phase, we cannot distinguish them.}}
  \label{fig:trace}
  \end{center}
\end{figure}

If $\hat R \gg 1$ and, more generally, if the chains were not mixing, this would be cause for concern and an invitation to adjust our inference method.
Even when $\hat R = 1$, we should entertain the possibility that all the chains suffer from the same bias.
Stan offers additional diagnostics to identify sampling biases, notably by reporting \textit{divergent transitions} of the HMC sampler, a topic we will discuss when we fit more sophisticated models.

\subsubsection{Controlling the variance of our estimator}

Let's assume that our samples are indeed unbiased.
The expected error of our estimator is now determined by the variance.
Under certain regularity conditions, our estimator follows an MCMC central limit theorem,
\begin{equation*}
  \hat {\mathbb E} f \overset{\mathrm{approx}}{\sim} \mathrm{Normal} \left ( \mathbb E f, \frac{\mathrm{Var} f}{n_\mathrm{eff}} \right ).
\end{equation*}
where the key difference is that $\mathrm{Var} f$ is now divided by the \textit{effective sample size}, $n_\mathrm{eff}$, rather than the sample size.
This change accounts for the fact our samples are not independent: their correlation induces a loss in information, which increase the variance of our estimator.
For $CL$, we have 2,000 samples, but the effective sample size is 1,580 (Table~\ref{tab:summary}).
If $n_\mathrm{eff}$ is low, our estimator may not be precise enough and we should generate more samples.

The effective sample size is only formally defined in the context of estimators for expectation values.
We may also be interested in tail quantities, such as extreme quantiles, which are much more difficult to estimate and require many more samples to achieve a desired precision.
\citet{Vehtari:2020} propose a generalization of the effective sample size for such quantities, and introduce the \textit{tail effective sample size}.
This is to be distinguished from the traditional effective sample size, henceforth the \textit{bulk effective sample size}.
Both quantities are reported by Stan.

\subsection{Checking the model: posterior predictive checks}  \label{sec:twoCpt_ppc}

Once we develop enough confidence in our inference, we still want to check our fitted model.
There are many ways of doing this.
We may look at the posterior distribution of an interpretable parameter and see if it suggests implausible values.
Or we may evaluate the model's ability to perform a certain task, e.g. classification or prediction, as is often done in machine learning.
In practice, we find it useful to do \textit{posterior predictive
  checks} (PPC), that is simulate data from the fitted model and compare the simulation to the observed data \cite[chapter 6]{Gelman:2013}.
%
Mechanically, the procedure is straightforward:
\begin{enumerate}
  \item Draw the parameters from their posterior, $\tilde \theta \sim p(\theta \mid y).$
  \item Draw new observations from the likelihood, conditional on the drawn parameters, $\tilde y \sim p(y \mid \tilde \theta)$.
\end{enumerate}
This amounts to drawing observations from their posterior distribution, that is $\tilde y \sim p(\tilde y \mid y)$.
The uncertainty due to our estimation and the uncertainty due to our measurement error are then propagated to our predictions.

Stan provides a \texttt{generated quantities} block, which allows us to compute values, based on sampled parameters.
In our two compartment model example, the following code draws new observations from the likelihood:
\begin{lstlisting}
generated quantities {
  real concentrationObsPred[nObs] 
    = lognormal_rng(log(concentrationHat[iObs]), sigma);
}
\end{lstlisting}
%
Here, we generated predictions at the observed points for each sampled point, $\theta^{(i)}$.
This gives us a sample of predictions and we can use the $5^\mathrm{th}$ and $95^\mathrm{th}$ quantiles to construct a credible interval.
We may then plot the observations and the credible intervals (Figure~\ref{fig:ppc}) and see that, indeed, the data generated by the model is consistent with the observations.
 
\begin{figure}
  \begin{center}
  \includegraphics[width=6in]{../figures/twocpt_ppc_4x8.pdf}
  \end{center}
  \caption{Posterior predictive checks for two compartment model. \textit{The circles represent the observed data and the shaded areas the $50^\mathrm{th}$ and $90^\mathrm{th}$ credible intervals based on posterior draws.}}
  \label{fig:ppc} 
\end{figure}

\subsection{Comparing models: leave-one-out cross validation}

Beyond model criticism, we may be interested in model comparison.
Continuing our running example, we compare our two compartment model to a one compartment model, which is also supported by Torsten via the \texttt{pmx\_solve\_onecpt} routine.
The corresponding posterior predictive checks are shown in Figure~\ref{fig:ppc_onecpt}.

\begin{figure}
  \begin{center}
  \includegraphics[width=6in]{../figures/onecpt_ppc_4x8.pdf}
  \end{center}
  \caption{Posterior predictive checks for one compartment model. \textit{The circles represent the observed data and the shaded areas the $50^\mathrm{th}$ and $90^\mathrm{th}$ credible intervals based on posterior draws. A graphical inspection suggests the credible interval is wider for the one compartment model than they are for the two compartment model.}}
  \label{fig:ppc_onecpt} 
\end{figure}

There are several ways of comparing models and which method is appropriate crucially depends on the insights we wish to gain.
If our goal is to asses a model's ability to make good out-of-sample predictions, we may consider \textit{Bayesian leave-one-out} (LOO) cross validation.
The premise of cross-validation is to exclude a point, $(y_i, x_i)$, from the \textit{training set}, i.e. the set of data to which we fit the model.
Here $x_i$ denotes the covariate and in our example, the relevant row in the event schedule.
We denote the reduced data set, $y_{-i}$.
We then generate a prediction $(\tilde y_i, x_i)$ using the fitted model, and compare $\tilde y_i$ to $y_i$.
A classic metric to make this comparison is the squared error, $(\tilde y_i - y_i)^2$.

Another approach is to use the \textit{LOO estimate of out-of-sample predictive fit}:
\begin{equation*}
  \mathrm{elp}_\mathrm{loo} := \sum_{i}^n \log p(y_i \mid y_{-i}).
\end{equation*}
%
Here, no prediction is made.
We however examine how consistent an ``unobserved'' data point is with our fitted model.
Computing this estimator is expensive, since it requires fitting the model to $n$ different training sets in order to evaluate each term in the sum.

\citet{Vehtari:2016} propose an estimator of $\mathrm{elp}_\mathrm{loo}$, which uses Pareto smooth importance sampling and only requires a single model fit.
The premise is to compute
\begin{equation*}
  \log p(y_i \mid y)
\end{equation*}
and correct this value, using importance sampling, to estimate $\log p(y_i \mid y_{-i})$.
Naturally this estimator may be inaccurate.
What makes this tool so useful is that we can use the Pareto shape parameter, $\hat k$, to asses how reliable the estimate is.
In particular, if $\hat k > 0.7$, then the estimate shouldn't be trusted.
The estimator is implemented in the R package Loo \cite{Gabry:2020}.

Conveniently, we can compute $\log p(y_i \mid y)$ in Stan's \texttt{generated quantities} block.
\begin{lstlisting}
vector[nObs] log_lik;
for (i in 1:nObs)
  log_lik[i] = 
   lognormal_lpdf(cObs[i] | log(concentrationHat[iObs]),
                  sigma);
\end{lstlisting}
These results can then be extracted and fed into Loo to compute $\mathrm{elp}_\mathrm{loo}$.
The file \texttt{twoCpt.r} in the Supplementary Material shows exactly how to do this.
Figure~\ref{fig:loo} plots the estimated $\mathrm{elp}_\mathrm{loo}$, along with a standard deviation, and shows the two compartment model has better out-of-sample predictive capabilities.

\begin{figure}
  \begin{center}
    \includegraphics[width = 5in]{../figures/elpd_loo_comp_4x8.pdf}
    \caption{Leave-one-out estimate of out-of-sample predictive fit. \textit{Plotted is the estimate, $\mathrm{elp}_\mathrm{loo}$, for the one and two compartment models. Clearly, the two compartment models has superior predictive capabilities.}}
    \label{fig:loo}
  \end{center}
\end{figure}




\section{Two compartment population model}

We now consider the scenario where we have data from multiple patients and fit a population model.
Population models are a powerful tool to capture the heterogeneity between patients, while also recognizing similarities.
Building the right prior allows us to pool information between patients, the idea being that what we learn from one patient teaches us something -- though not everything -- about the other patients.
In practice, such models can frustrate inference algorithms and need to be implemented with care \citep[e.g.][]{Betancourt:2013}.
We start with an example where the interaction between the model and our MCMC sampler is well behaved.
In Part II of this tutorial, we will examine a more difficult case, for which we will leverage Stan's diagnostic capabilities in order to run reliable inference.

\subsection{Statistical model} \label{sec:twoCptPop_stat}

Let $\vartheta$ be the 2D array of physiological parameters for each patient,
with
\begin{equation*}
  \vartheta_j = (CL_j, Q_j, ka_j, V_{\mathrm{cent}, j}, V_{\mathrm{peri}, j}, k_{a, j}),
\end{equation*} 
the parameters for the $j^\mathrm{th}$ patient.
We construct a population model by introducing the prior 
\begin{eqnarray*}
  \log \vartheta_j \sim \mathrm{Normal} (\log \vartheta_\mathrm{pop}, \Omega),
\end{eqnarray*}
for each patient.
As before we work on the log scale to account for the fact the physiological parameters are constrained to be positive.
$\vartheta_\mathrm{pop} = (CL_\mathrm{pop}, Q_\mathrm{pop}, V_\mathrm{cent, pop}, V_\mathrm{peri, pop}, k_\mathrm{a, pop})$ is the population mean and $\Omega$ the population covariance matrix.
Both $\vartheta_\mathrm{pop}$ and $\Omega$ are estimated.
% Prior information about the physiological parameters can now be encoded into priors on the population parameters, rather than on the individual parameters.
In this example, we start with the common case where $\Omega$ is diagonal.

The likelihood remains mostly unchanged, with the caveat that it must now be computed for each patient.
Putting this all together, we have the following model, as specified by the joint distribution,
\begin{eqnarray*}
  \vartheta_\mathrm{pop} & \sim & p(\vartheta_\mathrm{pop}), \hspace{1in} \text{(prior on physiological parameters)} \\
  \Omega & \sim & p(\Omega), \hspace{1.2in} \text{(prior on population covariance)} \\
  \sigma & \sim & p(\sigma) \\
  \vartheta \mid \vartheta_\mathrm{pop}, \Omega  & \sim  & \mathrm{logNormal}(\vartheta_\mathrm{pop}, \Omega), \\
  \log y \mid c, \sigma & \sim & \mathrm{Normal}(\log c, \sigma).
\end{eqnarray*}

\subsection{Specifying the model in Stan}

We begin by adjusting our parameters block:
\begin{lstlisting}[style=stan, numbers=none]
parameters {
  // Population parameters
  real<lower = 0> CL_pop;
  real<lower = 0> Q_pop;
  real<lower = 0> VC_pop;
  real<lower = 0> VP_pop;
  real<lower = 0> ka_pop;

  // Inter-individual variability
  vector<lower = 0>[5] omega;
  real<lower = 0> theta[nSubjects, 5];

  // measurement error
  real<lower = 0> sigma;
}
\end{lstlisting}
%
The variable, $\vartheta_\mathrm{pop}$ is introduced in \texttt{transformed parameters},
mostly for convenience purposes:
\begin{lstlisting}[style=stan, numbers=none]
vector<lower = 0>[nTheta]  
  theta_pop = to_vector({CL_pop, Q_pop, VC_pop, VP_pop, 
                         ka_pop});
\end{lstlisting}
%
The model block reflects our statistical formulation:
\begin{lstlisting}[style=stan, numbers=none] 
model {
  // prior on population parameters
  CL_pop ~ lognormal(log(10), 0.25); 
  Q_pop ~ lognormal(log(15), 0.5);
  VC_pop ~ lognormal(log(35), 0.25);
  VP_pop ~ lognormal(log(105), 0.5);
  ka_pop ~ lognormal(log(2.5), 1);
  omega ~ lognormal(0.25, 0.1);
  
  sigma ~ normal(0, 1);

  // hierarchical prior
  for (j in 1:nSubjects)
    theta[j, ] ~ lognormal(log(theta_pop), omega);

  // likelihood
  logCObs ~ normal(log(concentrationObs), sigma);
}
\end{lstlisting}
%
It remains to compute \texttt{concentrationObs}.
There are several ways to do this and, depending on the computational resources available, we may either compute the concentration for each patients sequentially or in parallel.
For now, we do the simpler sequential approach.
In the upcoming Part II of this tutorial, we examine how Torsten offers easy-to-use parallelization  for population models.

Sequentially computing the concentration is a simple matter of bookkeeping.
In \texttt{transformed parameters} we loop through the patients using a \texttt{for} loop.
The code is identical to what we used in Section~\ref{sec:twocpt_transformed_parameters},
with the caveat that the arguments to \texttt{pmx\_solve\_twocpt} are now indexed to indicate for which patient we compute the drug mass.
For example, assuming the time schedule is ordered by patient, the event times corresponding to the $j^\mathrm{th}$ patient are given by
\begin{lstlisting}[style=stan, numbers=none]
time[start[j]:end[j]]
\end{lstlisting}
%
where \texttt{start[j]} and \texttt{end[j]} contain the indices of the first and last event for the $j^\mathrm{th}$ patient, and the syntax for indexing is as in R.
The full \texttt{for} loop is then
%
\begin{lstlisting}[style=stan, numbers=none]
for (j in 1:nSubjects) {
  mass[, start[j]:end[j]] = 
    pmx_solve_twocpt(time[start[j]:end[j]],
                     amt[start[j]:end[j]],
                     rate[start[j]:end[j]],
                     ii[start[j]:end[j]],
                     evid[start[j]:end[j]],
                     cmt[start[j]:end[j]],
                     addl[start[j]:end[j]],
                     ss[start[j]:end[j]],
                     theta[j, ]);

  concentration[start[j]:end[j]] = 
         mass[2, start[j]:end[j]] / theta[j, 3];
  }
\end{lstlisting}

Once we have written our Stan model, we can apply the same methods for inference and diagnostics as we did in the previous section.

\subsection{Posterior predictive checks}

We follow the exact same procedure as in Section~\ref{sec:twoCpt_ppc} -- using even the same line of code -- to create new observations for our patients.
Figure~\ref{fig:twoCptPop_ppc} plots the results across patients.
In addition, we simulate data for new patients by: (i) drawing physiological parameters from our population distribution, (ii) solving the ODEs with these simulated parameters and (iii) using our measurement model to simulate new observations.
The generated quantities block then looks as follows:
%
\begin{lstlisting}[style=stan, numbers=none]
generated quantities {
  // Posterior predictive checks for existing patients
  real concentrationObsPred[nObs] 
    = exp(normal_rng(log(concentrationObs), sigma));

  // Posterior predictive checks for new patients
  // (here we assume they receive the same treatment 
  // as the observed patients)
  real cObsNewPred[nObs];
  matrix<lower = 0>[nCmt, nEvent] massNew;
  real thetaNew[nSubjects, nTheta];
  row_vector<lower = 0>[nEvent] concentrationNew;
  row_vector<lower = 0>[nObs] concentrationObsNew;

  for (j in 1:nSubjects) {
    // (i) simulate physiological parameters
    thetaNew[j, ] = lognormal_rng(log(theta_pop),omega);

    // (ii) solve ODEs and compute drug mass
    massNew[, start[j]:end[j]]
      = pmx_solve_twocpt(time[start[j]:end[j]],
                         amt[start[j]:end[j]],
                         rate[start[j]:end[j]],
                         ii[start[j]:end[j]],
                         evid[start[j]:end[j]],
                         cmt[start[j]:end[j]],
                         addl[start[j]:end[j]],
                         ss[start[j]:end[j]],
                         thetaNew[j, ]);

      concentrationNew[start[j]:end[j]]
        = massNew[2, start[j]:end[j]] / thetaNew[j, 3];

      concentrationObsNew = concentrationNew[iObs];
  }

  // (iii) simulate measurement error 
  cObsNewPred = exp(normal_rng(log(concentrationObsNew), sigma));
}
\end{lstlisting}

\begin{figure}
  \begin{center}
  \includegraphics[width=6in]{../figures/twocpt_pop_ppc_4x8.pdf}
  \end{center}
  \caption{Posterior predictive checks for poulation two compartment model.}
  \label{fig:twoCptPop_ppc}
\end{figure}

It is worth noting that the computational cost of running operations in the \texttt{generated quantities} is relatively small.
While these operations are executed once per iteration, in order to generate posterior samples of the generated quantities, operations in the \texttt{transformed parameters} and \texttt{model} blocks are run and differentiate multiple times per iterations, meaning they amply dominate the computation.
Hence the cost of doing posterior predictive checks, even when it involves solving ODEs, is marginal.
The computational scaling of Stan, notably for ODE-based models, is discussed in the article by \citet{Grinsztajn:2021}.


\section{Non-linear pharmacokinetic / pharmacodynamic model}

As a starting example, we consider a compartment pharmacokinetic model for a single patient.
The patient receives multiple doses at regular time intervals and the drug plasma concentration is recorded over time.
Our goal is to infer the physiological parameters of the patient, pertinent to the drug's pharmacokinetics, and the measurement error.

\subsection{Nonlinear pharmacokinetic / pharmacodynamic model} 
For the last example, let us go back to the single patient
two-compartment model and append it with a PD
model. Specifically, we examine the
Friberg-Karlsson semi-mechanistic model for drug-induced
myelosuppression [27, 28, 29, 30, 31, 14] with the goal to model the
relation between neutrophil counts and drug exposure.
It describes a delayed feedback mechanism that keeps the absolute neutrophil count (ANC) at the
baseline ($\text{Circ}_0$) in a circulatory compartment ($y_{\text{circ}}$), as well as the drug
effect that perturbs this meachanism. The delay between
proliferative cells ($y_{\text{prol}}$) and $y_{\text{circ}}$ is modeled by three
transit compartments with mean transit time
\begin{equation}
  \text{MTT} = (3 + 1)/k_{\text{tr}}
\end{equation}
where $k_{\text{tr}}$ is the transit rate constant.
\begin{figure}
  \begin{center}
  \includegraphics[width=5in]{../figures/neutrophilModel.jpg}
  \caption{Friberg-Karlsson semi-mechanistic Model.}
  \label{fig:fk_model}
  \end{center}
\end{figure}

The PD model can be summarized as
\begin{align*}
  \log(\text{ANC})& \sim \text{Normal}(\log(y_{\text{circ}}), \sigma^2_{\text{ANC}}),  \\
  y_{\text{circ}}& = f_{\text{FK}}(\text{MTT}, \text{Circ}_{0}, \alpha, \gamma, c),
  % (\text{MTT}, \text{Circ}_{0}, \alpha, \gamma, k_{\text{tr}})& = (125, 5.0, 3 \times 10^{-4}, 0.17) \\
  % \sigma^2_{\text{ANC}}& = 0.001
\end{align*}
where $c=y_{\text{cent}}/V_{\text{cent}}$ is the drug concentration calculated from the PK model, and function $f_{\text{FK}}$ represents solving  the following
nonlinear ODE for $y_{\text{circ}}$ 
\begin{subequations}\label{eq:FK}
\begin{align}
  \frac{dy_\mathrm{prol}}{dt} &= k_\mathrm{prol} y_\mathrm{prol} (1 - E_\mathrm{drug})\left(\frac{\text{Circ}_0}{y_\mathrm{circ}}\right)^\gamma - k_\mathrm{tr}y_\mathrm{prol}, \\
  \frac{dy_\mathrm{trans1}}{dt} &= k_\mathrm{tr} y_\mathrm{prol} - k_\mathrm{tr} y_\mathrm{trans1}, \\
  \frac{dy_\mathrm{trans2}}{dt} &= k_\mathrm{tr} y_\mathrm{trans1} - k_\mathrm{tr} y_\mathrm{trans2},  \\
  \frac{dy_\mathrm{trans3}}{dt} &= k_\mathrm{tr} y_\mathrm{trans2} - k_\mathrm{tr} y_\mathrm{trans3},  \\
  \frac{dy_\mathrm{circ}}{dt} &= k_\mathrm{tr} y_\mathrm{trans3} - k_\mathrm{tr} y_\mathrm{circ},
\end{align}
\end{subequations}
We use $E_{\text{drug}} = \alpha c$ to model the linear effect of drug
concentration in central compartment that reduces the proliferation rate or induces cell loss. The entire ODE system is formed by coupling equation \eqref{eq:twocpt}
and \eqref{eq:FK}.
The following \texttt{parameters} block summarizes the unknown parameters this model
\lstinputlisting[firstline=87,lastline=102]{../../Script/neutropenia-single-patient/neutropeniaSinglePatient1.stan}

Unlike in Section \ref{sec:twocpt}, here to solve the nonlinear system we must
utilize one of the numerical solvers in Torsten.

\subsubsection{Numerical solution of ODE}
To solve an ODE numerically in Stan we first need to define
its right-hand-side in the \texttt{functions} block, a block in which
we put all the user-supplied functions.
\lstinputlisting[firstline=1,lastline=38]{../../Script/neutropenia-single-patient/neutropeniaSinglePatient1.stan}
One can see that the above function is almost literal translation of
Eq. \eqref{eq:twocpt} and \eqref{eq:FK}, in that the first three
components of \texttt{dydt} describes the PK while the rest the
PD.

We omit most items in the \texttt{data} block but bring reader's attention to a
few items. It often helps to keep an index array that points to the
observations in the ODE solutions, such as
\lstinputlisting[firstline=50,lastline=51]{../../Script/neutropenia-single-patient/neutropeniaSinglePatient1.stan}
which are the indices to the the drug concentration and neutrophils
count observations, respectively. Despite Stan/Torsten provides
default values, we highly recommend user define the ODE
solver control parameters in the \texttt{data} block
\lstinputlisting[firstline=70,lastline=72]{../../Script/neutropenia-single-patient/neutropeniaSinglePatient1.stan}
and judiciously choose these control parameters
\begin{itemize}
\item \texttt{rtol}: relative tolerance to determine solution convergence,
\item \texttt{atol}: absolute tolerance to determine solution convergence,
\item \texttt{max\_num\_step}: maximum allowed steps.
\end{itemize}
In particular, user should make problem-dependent decision on \texttt{rtol} and \texttt{atol},
according to estimated scale of the unknowns, so that the error would
not affect inference on statistical variance of quantities that enter
the Stan model. For example, when an unknown can be neglected below
certain threshold without affecting the rest of the dynamic system,
setting \texttt{atol} greater than that threshold will avoid spurious and
error--prone computation. For more details, see \cite[Chapter 13]{Stan_users_guide:2021}
and \cite[Section 3.7.5]{Torsten:2021} and reference therein.

In \texttt{transformed data} block, we define fixed event specification
arguments such as \texttt{rate}, \texttt{ii}, etc, that are not
effective in this model. In fact here we also provide bioavailabity fraction \texttt{F} and dosing lag
time \texttt{tLag}, despite the fact that they are defined with default values
therefore can be omitted.
\lstinputlisting[firstline=75,lastline=85]{../../Script/neutropenia-single-patient/neutropeniaSinglePatient1.stan}

Now we are ready to solve the ODEs. Similar to Stan, Torsten provides
a few numerical solvers and in this
example we use the Runge-Kutta solver
\texttt{pmx\_solve\_rk45}\cite[Section 3.4]{Torsten:2021}. This
is done in the \texttt{transformed parameters} block.

\lstinputlisting[firstline=104,lastline=121]{../../Script/neutropenia-single-patient/neutropeniaSinglePatient1.stan}

One can see that \texttt{pmx\_solve\_rk45}
requires a user-defined ODE function as the first argument (similar to
how one solve an ODE using Stan function
\texttt{ode\_rk45}). In Stan functions like this are refered as
\emph{high--order functions} \cite[Chapter 9]{Stan:2021}.

\subsubsection{Solve PKPD model as coupled ODE system}
The approach in the last section applies to all the models that involve
ODE solutions, but we will not use it here. An acute
observer may have noticed the PKPD model here exhibits a particular
\emph{one-way coupling} structure.
That is, the PK (Eq. \eqref{eq:twocpt})
and PD (Eq. \eqref{eq:FK}) are
coupled through the proliferation cell count
$y_{\text{prol}}$ and $E_{\text{drug}}$ and the PK 
can be solved independently from the PD. This is what motivates Torsten's coupled solvers,
which analytically solves PK before
passing the PK solution to the PD and seeks its numerical
solution. Since the dimension of the numerical ODE solution is reduced, in general this coupled strategy is more efficient than
the last section's approach of numerically solving a full ODE system.
To see it in action, let us apply the
coupled solver \texttt{pmx\_solve\_twocpt\_rk45} \cite[Section 3.5]{Torsten:2021} to the same model. We need only make two changes. First, we
modify the ODE function to reflect that only the PD states are to be solved.
\lstinputlisting[firstline=1,lastline=30]{../../Script/neutropenia-single-patient/neutropeniaSinglePatientMix1.stan}

Note that here we pass in PK and PD states as separate arguments $y$
and $y_{\text{PK}}$, and the function describes the ODE for $y$, while
$y_{\text{PK}}$ will be solved internally using analytical solution,
so user do not need to explicitly call \texttt{pmx\_solve\_twocpt}.

Then we can simply replace \texttt{pmx\_solve\_rk45} with
\texttt{pmx\_solve\_twocpt\_rk45} call.
\lstinputlisting[firstline=108,lastline=108]{../../Script/neutropenia-single-patient/neutropeniaSinglePatientMix1.stan}

The \texttt{model} block is similar to that in Section \ref{sec:twocpt}.
\lstinputlisting[firstline=117,lastline=133]{../../Script/neutropenia-single-patient/neutropeniaSinglePatientMix1.stan}

\subsubsection{Posterior predictive checks}
We hope by now reader has developed the routine of performing
posterior predictive checks (PPC). Since we have both PK (drug
concentration) and PD (neutrophil count) observations, the PPC should
be conducted on both (Figure \ref{fig:neutro_ppc_1} and \ref{fig:neutro_ppc_2}).
\lstinputlisting[firstline=135,lastline=147]{../../Script/neutropenia-single-patient/neutropeniaSinglePatientMix1.stan}

\begin{figure}
  \begin{subfigure}[b]{0.45\textwidth}
    \includegraphics[width=\textwidth]{../figures/neutrophil_ppc_pk.pdf}
    \caption{PK: drug concentration}
    \label{fig:neutro_ppc_1}
  \end{subfigure}
  \qquad
  \begin{subfigure}[b]{0.45\textwidth}
    \includegraphics[width=\textwidth]{../figures/neutrophil_ppc_pd.pdf}
    \caption{PD: neutrophil count}
    \label{fig:neutro_ppc_2}
  \end{subfigure}
  \caption{Posterior predictive checks for the PKPD model}
\end{figure}


\section{Conclusion}
\input{conclusion.tex}

\bibliographystyle{abbrv}
\bibliography{../ref.bib}






\end{document}
