\documentclass[11pt]{article}
\usepackage{newclude}
% \usepackage{apacite}
\usepackage{amssymb,amsmath}
\usepackage{color}
\usepackage{graphicx}
\usepackage{natbib}
\usepackage{subcaption} 

% TODO: change options
\usepackage{hyperref}

% Verbantim - use to define \texttt{Stan} Code
\usepackage{fancyvrb}

% Code and packages for code figures
\usepackage{pdfpages}
\usepackage{listings}
\newsavebox{\fmbox}
\newenvironment{fmpage}[1]
   {\begin{lrbox}{\fmbox}\begin{minipage}{#1}}
   {\end{minipage}\end{lrbox}\fbox{\usebox{\fmbox}}}

% enumerate using letters
\usepackage{enumitem}

% package to control position of figures
\usepackage{float}

% control margins
\usepackage[left=4cm, right=4cm, top=2cm]{geometry}

% for fancy Mike style table
\usepackage[table]{xcolor} 
\usepackage{multirow}

% algorithm
\usepackage{algorithm, setspace}
\usepackage[noend]{algpseudocode}
\DefineVerbatimEnvironment{Stancode}{Verbatim}{fontsize=\small,xleftmargin=2em}  % \texttt{Stan} Code

% package for drawing graphs
\usepackage{tikz}
\usetikzlibrary{positioning}

% quote
\usepackage{epigraph}

% Code to write Stan code
\usepackage{listings}
\lstset{escapeinside={<@}{@>}}
\definecolor{codegreen}{rgb}{0,0.6,0}
\definecolor{codegray}{rgb}{0.5,0.5,0.5}
\definecolor{codepurple}{rgb}{0.58,0,0.82}
\definecolor{backcolour}{rgb}{0.99,0.99,0.97}
\lstdefinestyle{stan}{
	literate={~}{$\sim$}{1},
	backgroundcolor=\color{backcolour}, 
	comment=[l]{//},
	commentstyle=\color{codegreen},
	keywords = {int, real, vector, matrix, data, model, parameters, transformed, target, generated, quantities, for, in},
	keywordstyle=\color{magenta},
	numberstyle=\tiny\color{codegray},
	stringstyle=\color{codepurple},
	emph={%  
		normal, normal_rng,
		pmx_solve_twocpt,
		cmdstan_model,
		exp, log
	},
	emphstyle=\color{codepurple},%
	basicstyle={\ttfamily},
	% breakatwhitespace=false,         
	breaklines=true,                 
	% captionpos=b,                    
	keepspaces=true,                 
%	numbers=left,                    
%	numbersep=5pt,                  
	showspaces=false,                
	% showstringspaces=false,
	% showtabs=false,                  
	% tabsize=2
}
\lstset{style=stan}

\title{Bayesian pharmacometrics modeling \\ using Stan and Torsten, Part I}
\author{}
\date{}

\begin{document}

\maketitle

\begin{abstract}
  Stan is an open-source probabilistic programing language, primarily designed to do Bayesian data analysis.
  Its main inference algorithm is an adaptive Hamiltonian Monte Carlo sampler, supported by state of the art gradient computation.
  Stan bolsters several strengths, notably efficient computation, an expressive language which offers a great deal of flexibility, and numerous diagnostics that allow modelers to check whether the inference is reliable.
  % Over the past years, much development efforts have been geared towards supporting models based on differential equations, such as those that arise in pharmacometrics.
  Torsten extends Stan with a suite of functions that facilitate the specification of pharmacokinetic and pharmacodynamic models, and makes it straightforward to specify a clinical event schedule.
  Part I of this tutorial demonstrates how to build, fit, and criticize simple pharmacokinetic models using Stan and Torsten.
\end{abstract}

\section{Introduction}
\input{Intro.tex}
% Basic concepts of Bayesian modeling
% Why Stan?
% Why Torsten?
% Downloading the software and references.

%\section{Specifying a model in Stan}
%\input{basicStan.tex}

\section{Two compartment model}  \label{sec:twoCpt}



\section{Two compartment population model}



\section{Nonlinear pharmacokinetic / pharmacodynamic model}



\section{Conclusion}



\bibliographystyle{plain}
\bibliography{../ref}


\end{document}
